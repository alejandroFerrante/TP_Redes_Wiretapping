\section{Implementación}
La herramienta se encuentra implementada en el archivo \texttt{traceroute.py}. Toma como parámetros la dirección a envíar los paquetes, el tamaño de las ráfagas de paquetes (\texttt{burst\_size}) y el máximo TTL.

Para cada TTL entre 1 y el máximo se envían \texttt{burst\_size} paquetes con el TTL adecuado y se reciben las respuestas, usando la función \texttt{sr()} de \texttt{scapy}. Si se recibe un echo reply, se dejan de mandar ráfagas.

A partir de los paquetes recibidos como respuesta de un TTL se obtiene una IP (la más común de entre los hosts que respondieron) y un RTT (el promedio de RTT para la IP elegida). Si no se recibe ninguna respuesta, ambos son nulos. Elegimos promediar sólo los RTT de los saltos con la IP elegida y no todos porque esto es más consistente con la elección.

Una vez procesados los envíos, empieza el análisis para predecir los saltos intercontinentales. Aplicamos el método de Cimbala para obtener los outliers, que serán los saltos intercontinentales predichos. Los datos a analizar son las diferencias de RTT en cada salto con el salto anterior no nulo. 

Los saltos con menor RTT que el inmediatamente anterior tienen una diferencia negativa, y decidimos ignorarlos en el análisis ya que no representan magnitudes reales.

%% podemos mandarle el pseudocodigo de cimbala o no ... no se P:
