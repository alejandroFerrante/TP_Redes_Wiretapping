\documentclass[10pt, a4paper]{article}
\usepackage[paper=a4paper, left=1.5cm, right=1.5cm, bottom=1.5cm, top=2cm]{geometry}
\usepackage[utf8]{inputenc}
\usepackage[T1]{fontenc}
\usepackage[spanish]{babel}
\usepackage{indentfirst}
\usepackage{fancyhdr}
\usepackage{lastpage}
\usepackage{calc}
\usepackage{caratula}
\usepackage{marvosym} % para \Faxmachine !
\usepackage{graphicx}
\usepackage{float}
% \PassOptionsToPackage{noend}{algpseudocode}% comment out if want end's to show
\usepackage{algpseudocode}
\usepackage{algorithm}
\usepackage{multicol}
\usepackage[hidelinks]{hyperref}
\graphicspath{{imagenes/}}
%\sloppy
\parskip=5pt % 10pt es el tamano de fuente

\usepackage{stringenc}
\usepackage{pdfescape}
\usepackage{color}
\definecolor{red}{RGB}{255,0,0}
\definecolor{blue}{RGB}{0,0,255}
\usepackage{amsmath}
\usepackage[makeroom]{cancel}
\usepackage{wrapfig}
\usepackage[font=small,labelfont=bf]{caption}


%% -------------------------
\errorcontextlines\maxdimen

% begin vertical rule patch for algorithmicx (http://tex.stackexchange.com/questions/144840/vertical-loop-block-lines-in-algorithmicx-with-noend-option)
\makeatletter
% start with some helper code
% This is the vertical rule that is inserted
    \newcommand*{\algrule}[1][\algorithmicindent]{\makebox[#1][l]{\hspace*{.5em}\thealgruleextra\vrule height \thealgruleheight depth \thealgruledepth}}%
% its height and depth need to be adjustable
\newcommand*{\thealgruleextra}{}
\newcommand*{\thealgruleheight}{.75\baselineskip}
\newcommand*{\thealgruledepth}{.25\baselineskip}

\newcount\ALG@printindent@tempcnta
\def\ALG@printindent{%
    \ifnum \theALG@nested>0% is there anything to print
        \ifx\ALG@text\ALG@x@notext% is this an end group without any text?
            % do nothing
        \else
            \unskip
            \addvspace{-1pt}% FUDGE to make the rules line up
            % draw a rule for each indent level
            \ALG@printindent@tempcnta=1
            \loop
                \algrule[\csname ALG@ind@\the\ALG@printindent@tempcnta\endcsname]%
                \advance \ALG@printindent@tempcnta 1
            \ifnum \ALG@printindent@tempcnta<\numexpr\theALG@nested+1\relax% can't do <=, so add one to RHS and use < instead
            \repeat
        \fi
    \fi
    }%
\usepackage{etoolbox}
% the following line injects our new indent handling code in place of the default spacing
\patchcmd{\ALG@doentity}{\noindent\hskip\ALG@tlm}{\ALG@printindent}{}{\errmessage{failed to patch}}
\makeatother

% the required height and depth are set by measuring the content to be shown
% this means that the content is processed twice
\newbox\statebox
\newcommand{\myState}[1]{%
    \setbox\statebox=\vbox{#1}%
    \edef\thealgruleheight{\dimexpr \the\ht\statebox+1pt\relax}%
    \edef\thealgruledepth{\dimexpr \the\dp\statebox+1pt\relax}%
    \ifdim\thealgruleheight<.75\baselineskip
        \def\thealgruleheight{\dimexpr .75\baselineskip+1pt\relax}%
    \fi
    \ifdim\thealgruledepth<.25\baselineskip
        \def\thealgruledepth{\dimexpr .25\baselineskip+1pt\relax}%
    \fi
    %\showboxdepth=100
    %\showboxbreadth=100
    %\showbox\statebox
    \State #1%
    %\State \usebox\statebox
    %\State \unvbox\statebox
    %reset in case the next command is not wrapped in \myState
    \def\thealgruleheight{\dimexpr .75\baselineskip+1pt\relax}%
    \def\thealgruledepth{\dimexpr .25\baselineskip+1pt\relax}%
}
% end vertical rule patch for algorithmicx
%%--------------------------



\begin{document}
\title{TDC - TP2}
\materia{Teoría de las Comunicaciones}
\submateria{Segundo cuatrimestre 2017}
\titulo{Grupo 6}
\begin{center}
    \includegraphics[width=0.7\textwidth]{caratula.jpg}
\end{center}
\subtitulo{TP2}
\integrante{Alejandro Ferrante}{371/09}{matapalabras@hotmail.com}
\integrante{Gonzalo Guillamon}{97/12}{gonzaguillamon@gmail.com}
\integrante{Malena Ivnisky}{421/12}{malenaivnisky@gmail.com}

\maketitle

% \newpage\null\thispagestyle{empty}

\newpage
\thispagestyle{empty}
\setcounter{tocdepth}{3}
\tableofcontents

% \newpage\null\thispagestyle{empty}

\newpage
\setcounter{page}{1}


%\includegraphics[width=\textwidth]{cookies}  ejemplo para incluir imagenes

\section{Resumen}

En este trabajo práctico realizamos mediciones sobre tres redes diferentes, dos domésticas y una pública. Las analizamos en función de dos modelos de fuentes, y vimos las redes de mensajes ARP subyacentes, que eran muy simples en el caso de las redes domésticas y muy complicada en el caso de la red pública. En general vimos los protocolos que esperábamos ver y uno que no conocíamos. Vimos muchas similitudes entre las dos redes domésticas.

\section{Introducción}

El objetivo de este trabajo práctico es analizar distintas capturas de red usando las herramientas vistas en la materia.
\section{Implementación}
La herramienta se encuentra implementada en el archivo \texttt{traceroute.py}. Toma como parámetros la dirección a envíar los paquetes, el tamaño de las ráfagas de paquetes (\texttt{burst\_size}) y el máximo TTL.

Para cada TTL entre 1 y el máximo se envían \texttt{burst\_size} paquetes con el TTL adecuado y se reciben las respuestas, usando la función \texttt{sr()} de \texttt{scapy}. Si se recibe un echo reply, se dejan de mandar ráfagas.

A partir de los paquetes recibidos como respuesta de un TTL se obtiene una IP (la más común de entre los hosts que respondieron) y un RTT (el promedio de RTT para la IP elegida). Si no se recibe ninguna respuesta, ambos son nulos. Elegimos promediar sólo los RTT de los saltos con la IP elegida y no todos porque esto es más consistente con la elección.

Una vez procesados los envíos, empieza el análisis para predecir los saltos intercontinentales. Usamos como datos las diferencias de los RTT promedio entre cada salto y su anterior inmediato que no sea nulo. El primer salto no nulo tiene diferencia igual a su RTT promedio. Elegimos estos valores como datos a analizar porque esperamos que los saltos intercontinentales tarden más tiempo que los que no lo son, entonces un outlier para estos datos es un posible salto intercontinental.

Separamos los datos entre los saltos nulos (y con salto intercontinental, pero todavía no hay ninguno) y los demás saltos, que van a participar como datos válidos para encontrar a los outliers.

Aplicamos el método de Cimbala: iterativamente calculamos $\tau$, la media y la desviación estándar de las diferencias de RTT, y chequeamos si el salto con mayor distancia a la media es un outlier. Si es un outlier, lo marcamos como que es un salto intercontinental, lo movemos con los otros datos y seguimos. Sino, de acuerdo al método ya terminamos.

Finalmente volvemos a juntar los datos y los devolvemos por la salida estándar.

\section{Ruta 1: Universidad de Tokyo}

Ejecutamos la herramienta usando la dirección \texttt{www.u-tokyo.ac.jp}, con ráfagas de 100 paquetes y TTL máximo igual a 30. Los resultados obtenidos se muestran en las figuras \ref{tabla1} y \ref{mapa1}. Las ubicaciones indicadas fueron obtenidas usando una página de geolocalización de IP \cite{ip2location}.

\subsection{Análisis de la ruta obtenida}
Observamos que la ruta parte de Buenoa Aires como corresponde. Luego de no obtener respuesta en varios saltos, encontramos una IP supuestamente localizada en Italia. En el salto siguiente la IP se encuentra en Brasil.

Esto nos hizo sospechar acerca de la correctitud de esta ruta, ya que existen caminos directos entre Argentina y Brasil; no hay necesidad de pasar por Italia. Esto se puede ver en un mapa de cables submarinos \cite{cables}.

Como vimos que varias páginas de geolocalización de IP daban resultados distintos para la misma IP, usando una página que da resultados de fuentes distintas \cite{iplocation} comparamos resultados para la IP \texttt{195.22.219.3}. Obtuvimos como resultado que casi todas las fuentes devuelven una ubicación en Italia, mientras que una devuelve la ciudad de Fortaleza, en Brasil. Observamos que el cable submarino Altantis-2 conecta entre otras ciudades Las Toninas con Fortaleza \cite{atlantis2}. Además, figura como uno de los dueños del cable la empresa Telecom Italia Sparkle, que figura como ISP en varias búsquedas que ubican esta IP en Italia. Por esto creemos que la verdadera ruta va directamente hacia Fortaleza usando el cable Atlantis-2, ya que los métodos de geolocalización de IPs no son muy confiables \cite{accuracy}, y podría ocurrir que la ubicación de una IP situada físicamente en Brasil pero vinculada a una ISP italiana sea medida incorrectamente.

Luego la ruta sigue por San Pablo, y luego salta hasta Nueva York. En este caso mirando el mapa \cite{cables} vemos que se usa el cable Seabras-1, que une estas dos ciudades \cite{seabras1}.

Siguiendo por Estados Unidos, la ruta pasa por Seattle y salta hasta Osaka, en Japón. Esto es posible si se usa el cable Pacific Crossing-1, que entre otras ciudades une Shima, Japón (cercana a Osaka) con la zona de Seattle \cite{pc1}.

Finalmente se llega al destino en Tokyo.

\subsection{Análisis de las predicciones de salto intercontinental}

En la tabla de la figura \ref{tabla1} vemos que la mayoría de los saltos son correctamente predichos. Sólo $\frac{3}{17}$ resultan incorrectos, en total.

\begin{itemize}
	\item Porcentaje de saltos que no responden: 15\%
	\item Largo de la ruta de saltos que responden: 17 saltos 
	\item Cantidad de enlaces intercontinentales (separando América del Sur/Norte): 2
	\item Cantidad de outliers: 3
	\item Falsos positivos: 2
	\item Falsos negativos: 1
\end{itemize}

\begin{figure}[H]
\centering
\begin{tabular}{l | l | l | l | c | c}
Hop & RTT & IP & Ubicación & Predicción de SI & ¿correcto?\\
\hline
1 & 0.0017 & \texttt{192.168.10.1} & Buenos Aires, Argentina & false & \cmark\\
2 & 0.0051 & \texttt{192.168.0.1} & Buenos Aires, Argentina & false & \cmark\\
3 & 0.0258 & \texttt{10.31.0.1} & Buenos Aires, Argentina & false & \cmark\\
4 & 0.0185 & \texttt{10.242.2.133} & Buenos Aires, Argentina & false & \cmark\\
5 & 0.0178 & \texttt{195.22.220.33} & Buenos Aires, Argentina & false & \cmark\\
6 & 0.0189 & \texttt{195.22.220.32} & Buenos Aires, Argentina & false & \cmark\\
7 & 0.0435 & \texttt{195.22.219.17} & Fortaleza, Brasil/Italia & false & \cmark\\
8 & 0.1458 & \texttt{149.3.181.65} & San Pablo, Brasil & true & \xmark\\
9 & 0.1591 & \texttt{129.250.2.227} & Nueva York, EEUU & false & \xmark\\
10 & 0.2012 & \texttt{129.250.4.13} & Seattle, EEUU & true & \xmark\\
11 & 0.2032 & \texttt{129.250.2.54} & Seattle, EEUU & false & \cmark\\
12 & 0.3121 & \texttt{129.250.3.61} & Osaka, Japón & true & \cmark\\
13 & 0.3071 & \texttt{129.250.5.39} & Osaka, Japón & false & \cmark\\
14 & 0.3099 & \texttt{129.250.3.232} & Osaka, Japón & false & \cmark\\
15 & 0.3081 & \texttt{61.200.80.218} & Osaka, Japón & false & \cmark\\
16 & null & null & null & null \\
17 & null & null & null & null \\
18 & null & null & null & null \\
19 & 0.3251 & \texttt{154.34.240.254} & Tokyo, Japón & false & \cmark\\
20 & 0.3226 & \texttt{210.152.135.178} & Tokyo, Japón & false & \cmark\\
\end{tabular}
\caption{Tabla de resultados para la Universidad de Tokyo.}
\label{tabla1}
\end{figure}

\begin{figure}[H]
\includegraphics[width=\textwidth]{tokyo.png}
\caption{Mapa de resultados para la Universidad de Tokyo.}
\label{mapa1}
\end{figure}

En la figura \ref{diff1} se puede ver que la mayor diferencia se da en el salto 16, pero a la vez el mayor salto negativo se da en el salto 17. De esto deducimos que probablemente haya varios saltos posibles en esta zona, y la ruta elegida por la herramienta es una mezcla de alguna de estas. Esto ocurre cuando los paquetes están en Japón. El resto de los puntos no muestran una tendencia definida (ni creciente ni decreciente).

\begin{figure}[H]
\centering
\includegraphics[width=0.6\textwidth]{tokyo1.png}
\caption{Gráfico de diferencias de RTT en función de cada salto.}
\label{diff1}
\end{figure}

De la figura \ref{sdev1} vemos que los puntos están muy dispersos, lejos del 0 de $x$ (que representa el promedio). Por eso el método de Cimbala resultó con muchos outliers.

\begin{figure}[H]
\centering
\includegraphics[width=0.6\textwidth]{tokyo2.png}
\caption{Gráfico de $\frac{X_i - \bar{X}}{S}$ en función de las diferencias de RTT.}
\label{sdev1}
\end{figure}

Atribuimos los falsos positivos en Argentina a la diferencia de la velocidad de conexión con el resto del mundo. De acuerdo al reporte del estado de internet de Akamai \cite{akamai} nuestro país tiene baja velocidad promedio de internet, como se muestra en la figura \ref{speed}. Como estamos usando los RTT para calcular los saltos, no podemos saber si un RTT entre saltos más grande se debe a mayor distancia o a menor velocidad. Esto posibilita los falsos positivos.

En cuanto a Japón, vemos en la figura \ref{speed} que su velocidad de conexión es mucho más alta que el promedio. Pero como fue mencionado antes, en esa zona probablemente haya rutas alternativas internas y el criterio elegido puede mezclarlas entre sí, causando falsos positivos.

\begin{figure}[H]
\centering
\includegraphics[width=0.7\textwidth]{speed.png}
\caption{Velocidades promedio de internet en el mundo.}
\label{speed}
\end{figure}

\section{Ruta 2: Universidad de Ghana}

www.ug.edu.gh

\begin{tabular}{l | l | l | l | l}
Hop & RTT & IP & Ubicación & Salto Intercontinental\\
1 & 0.0231 & 192.168.0.1 & Buenos Aires, Argentina & true\\
2 & null & null & null & null\\
3 & null & null & null & null\\
4 & null & null & null & null\\
5 & null & null & null & null\\
6 & null & null & null & null\\
7 & 0.2930 & \texttt{200.89.165.130} & Buenos Aires, Argentina & true\\
8 & 0.1588 & \texttt{200.89.165.222} & Buenos Aires, Argentina & true\\
9 & 0.1693 & \texttt{185.70.203.32} & Buenos Aires, Argentina & true\\
10 & 0.2927 & \texttt{195.22.199.185} & Ashburn, EEUU & true\\
11 & 0.1995 & \texttt{216.6.87.202} & Ashburn, EEUU & true\\
12 & 0.6590 & \texttt{216.6.87.105} & Londres, Inglaterra & true\\
13 & 0.3755 & \texttt{80.231.130.130} & Londres, Inglaterra & true\\
14 & 0.4624 & \texttt{80.231.76.85} & Londres, Inglaterra & false\\
15 & 1.0908 & \texttt{195.219.195.42} & Singapur, Singapur & true\\
16 & 0.3648 & \texttt{41.204.60.149} & Kumasi, Ghana & true\\
17 & 0.4641 & \texttt{41.204.60.150} & Kumasi, Ghana & false\\
18 & 0.3928 & \texttt{197.255.127.6} & Accra, Ghana & true\\
19 & 0.7883 & \texttt{197.255.127.35} & Accra, Ghana & true\\
10 & 0.4128 & \texttt{197.255.125.213} & Accra, Ghana & true\\
\end{tabular}

\includegraphics[width=\textwidth]{ghana.png}
\section{Ruta 3: Universidad de Ljubljana}
Los paquetes hacen varios saltos en Buenos Aires (aunque algunos no responden podemos suponer que son en Buenos Aires), luego de lo cual hay un salto de Buenos Aires a Viena. Esto es anormal, ya que el unico cable submarino que conecta Argentina y Europa llega a España. Esto puede deberse a algún tipo de conexión directa entre Viena y dicho cable. Luego hay algunos saltos mas en Europa hasta llegar a destino. Hay un solo salto intercontinental en el mapa, que es el salto Buenos Aires-Viena, este salto es detectado correctamente, pero al igual que en experimentos anteriores hay muchos falsos positivos.


www.uni-lj.si

\begin{tabular}{l | l | l | l | l}
Hop & RTT & IP & Ubicación & Salto Intercontinental\\
1 & 0.0457 & \texttt{192.168.0.1} & Buenos Aires, Argentina & false\\
2 & null & null & null & null\\
3 & null & null & null & null\\
4 & null & null & null & null\\
5 & null & null & null & null\\
6 & 0.1651 & \texttt{200.89.161.81} & Buenos Aires, Argentina & true\\
7 & 0.3778 & \texttt{200.89.165.86} & Buenos Aires, Argentina & true\\
8 & 0.2096 & \texttt{185.70.203.22} & Buenos Aires, Argentina & true\\
9 & 0.3915 & \texttt{195.22.215.168} & Viena, Austria & true\\
10 & 0.4367 & \texttt{195.22.215.199} & Viena, Austria & false\\
11 & 0.5828 & \texttt{77.94.128.25} & Ljubljana, Eslovenia & true\\
12 & 0.3733 & \texttt{77.94.139.210} & Ljubljana, Eslovenia & true\\
13 & 0.4521 & \texttt{91.216.54.245} & Nova Gorica, Eslovenia & true\\
14 & 0.3534 & \texttt{91.223.115.153} & Nova Gorica, Eslovenia & true\\
\end{tabular}

\includegraphics[width=\textwidth]{ljubljana.png}
\section{Conclusiones}

No vimos falsos negativos. En general hay outliers de más, no de menos. 

Este método no es muy efectivo porque las velocidades de internet en cada país no son las mismas, y eso no es tenido en cuenta.
\begin{thebibliography}{99}

\bibitem{ip2location}
  \texttt{https://www.ip2location.com/}

\bibitem{cables}
	\texttt{https://www.submarinecablemap.com/}

\bibitem{iplocation}
	\texttt{https://www.iplocation.net/}

\bibitem{atlantis2}
	\texttt{https://en.wikipedia.org/wiki/Atlantis-2}

\bibitem{accuracy}
	\texttt{https://www.iplocation.net/geolocation-accuracy}

\bibitem{seabras1}
	\texttt{https://en.wikipedia.org/wiki/Seaborn\_Networks}

\bibitem{pc1}
	\texttt{https://en.wikipedia.org/wiki/PC-1}

\bibitem{seamewe}
	\texttt{https://en.wikipedia.org/wiki/SEA-ME-WE\_3}

\bibitem{wacs}
	\texttt{https://en.wikipedia.org/wiki/WACS\_(cable\_system)}

\bibitem{glo1}
	\texttt{https://en.wikipedia.org/wiki/GLO-1}

\end{thebibliography}

\end{document}
