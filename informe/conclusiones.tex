\section{Conclusiones}
Lo primero que pudimos observar, es que en todos los experimentos había una proporción muy grande de paquetes <unicast, ipv4>. Esto se debe a que en todos los casos, el uso principal de la red es acceder a internet, de ahí el gran intercambio de paquetes de este tipo. Pero otra parte pudimos observar que la proporción de paquetes broadcast en todos los casos fue menor a la de paquetes unicast. Esto puede deberse a que los paquetes broadcast se utilizan para obtener información acerca de la red o para informar cambios en la misma. Ya que los cambios no son frecuentes (al menos en los casos analizados) y, en consecuencia, la información que se obtiene de la red puede ser reutilizada, podemos concluir que no hay mucha necesidad de utilizar muchos paquetes de tipo broadcast.\\
Por último, pudimos observar un comportamiento muy similar en las dos redes hogareñas, con la excepción de que la red conectada por cable detectaba menos paquetes ARP. Esto puede deberse a que lo que se observa desde una conección wifi tiene una naturaleza mas "volatil" (en todo momento hay dispositivos conectandose y desconectandose de la red) por lo que es normal que se hayan observado mas paquetes de este tipo.