\section{Conclusiones}

Lo primero que hemos notado es lo cuidadosos que debemos ser al diseñar una fuente de Informacion y sobre todo la manera de agrupar y/o distinguir los simbolos, ya que estas decisiones pueden resultar en sets de datos muy diferentes.La fuente S1 sirvio para agrupar por protocoles, mientras que la S2 para distinguir direcciones de quienes mandaban requests ARP.
En cuanto a como cambia los resultados una desicion al diseñar la fuente, tenemos por ejemplo que en el tercer experimento dos direcciones que mandaron respuestas a las requests ARP y por tanto no aparecian como simbolos de S2. Si hubieramos diseñado S2 de tal manera que las respuestas a requests ARP tambien son consideradas simbolos, contariamos con sets de datos distintos y la entropia se hubiese visto modificada.

Observamos tambien que la entropia funciona a nivel practico como una cota bajo la cual se encuentran los simbolos mas relevantes como se ve claramente en el segundo experimento donde lo central del analisis se refiere exclusivamente a los simbolos que se encuentran por debajo de la entropia.

Otra cosa interesante de ver fue la brecha entre la entropia muestral y la maxima, ya que nos resulto una manera sencilla de cuantificar lo bueno o malo del diseño de una fuente, o lo atipico de una medicion en funcion a cuanto se acercaba a la entropia maxima. Dicho en otras palabras, la entropia maxima representa la entropia para una fuente equiprobable (donde la probabilidad de cada simbolo es uniforme). 

De los tres experimentos el mas interesante parecio ser el primero, ya que se trataba de una red abierta inalambrica con gran cantidad de dispositivos conectados a la misma. Por eso presento mayor variabilidad de datos y un analisis mas rico.

En general concluimos que la teoria de la informacion es una herramienta util para el analisis ya que permite ordenar claramente los datos y agruparlos adecuadamente para el analisis que se quiera realizar, ademas de proveer calculos y resultados teoricos como la entropia para poder sacar conclusiones.