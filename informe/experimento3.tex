\section{Experimento 3: Red hogareña cableada}

\subsection{Descripción del contexto}

El experimento fue realizado en una red doméstica, por medio de una conexión Wi-Fi. Al momento de tomar las mediciones estaban conectados a la red varias laptops y celulares. La fecha de la captura fue Domingo 8 de Octubre de 2017.

\subsection{Descripción de la captura}

Capturamos 11000 paquetes. En la figura~\ref{protocolos4} se muestra la distribución de los protocolos en la red. Observamos que la mayoría de los paquetes son de tipo IPv4, y muchos menos son de tipo ARP y \texttt{0x86dd} (IPv6).

\begin{figure}[H]
\centering
\includegraphics[width=0.7\textwidth]{protocolosRed4.png}
\caption{Gráfico que muestra la distribución de protocolos en la red.}
\label{protocolos4}
\end{figure}

En la figura~\ref{broadcast4} podemos ver el porcentaje de paquetes broadcast comparado con el total de paquetes. Vemos que esto representa un 2,2\% del total. Además, en la figura~\ref{entropias1_4} vemos que los paquetes de broadcast son de tipo IPv4 y ARP.

Este comportamiento es similar al del segundo experimento, ya que ambas son redes domésticas con pocos dispositivos conectados.

\begin{figure}[H]
\centering
\includegraphics[width=0.7\textwidth]{broadcastRed4.png}
\caption{Gráfico que muestra los porcentajes de tráfico broadcast y unicast.}
\label{broadcast4}
\end{figure}

\subsection{Análisis de la captura}

En la figura~\ref{entropias1_4} se muestra la información de cada símbolo de la fuente $S_1$. Hay un símbolo con mucha menor información que los demás (IPv4, unicast), se debe a que la mayoría de los paquetes fueron de este tipo. A causa de esto, la entropía de la fuente es muy baja, está muy lejos del máximo, porque hay grandes diferencias entre el símbolo de menor información y los demás.

\begin{figure}[H]
\centering
\includegraphics[width=0.7\textwidth]{entropiaS1Red4.png}
\caption{Gráfico de la información de los símbolos de la fuente $S_1$ observados en esta red. Se muestra la entropía muestral de $S_1$ y su entropía máxima.}
\label{entropias1_4}
\end{figure}

En cuanto a la fuente $S_2$, vemos en la figura ~\ref{entropias2_4} que sólo hay dos símbolos. La red es tan chica que hay solo dos IPs que realizaron request de ARP. La entropía muestral está cerca de $\frac{1}{2}$, la mitad que la entropía máxima.

\begin{figure}[H]
\centering
\includegraphics[width=0.7\textwidth]{entropiaS2Red4.png}
\caption{Gráfico de la información de los símbolos de la fuente $S_2$ observados en esta red. Se muestra la entropía muestral de $S_2$ y su entropía máxima.}
\label{entropias2_4}
\end{figure}

Por último graficamos la red subyacente de mensajes ARP, al igual que en los otros dos experimentos. Observamos que el grafo no es conexo, y no podemos identificar al router. Esto es consistente con el hecho de que no se midieron muchos paquetes ARP en total en esta red.

\begin{figure}[H]
\centering
\includegraphics[width=0.6\textwidth]{grafoRed4.png}
\caption{Grafo de la red de mensajes ARP subyacente.}
\label{grafo4}
\end{figure}
